\documentclass{beamer}
\usetheme{Madrid}
 
\usepackage[utf8]{inputenc}

\usepackage{amsmath}
\usepackage{amsfonts}
\usepackage{array}

\usepackage{graphicx}
\graphicspath{{./figs/}}
\usepackage{subfigure}
\usepackage{multirow}
\usepackage{xspace}


\usepackage{tikz}
\usetikzlibrary{arrows, shapes, positioning}



%Information to be included in the title page:
\title{Properties of consensus protocols}

\author{Runchao Han}
\institute{}
\date{}

\begin{document}

\frame{\titlepage}

%%%%%%%%%%%%%%%%%%%%%%%%%%%%%%%%%%%%%%%%%%%%%%%%%%%%%%%%%%%%%%%%%%%%%%%%%%%%%%%%%%%%%%%%%%%%%%%%

\begin{frame}
    \frametitle{Summary}


    \tikzstyle{block} = [draw, rectangle]
    \tikzstyle{property} = [draw, ellipse, font=\tiny]


    \begin{tikzpicture}[]
        \node [block] (consensus) {Consensus};
        \node [block, above left=2cm of consensus, align=center] (smr) {State \\ machine \\ replication};
        \node [block, above right=2cm of consensus] (blockchain) {Blockchain};

        \node [property, below=of consensus] (termination) {Termination};
        \node [property, below=of consensus, left=0.1cm of termination] (validity) {Validity};
        \node [property, below=of consensus, left=0.1cm of validity] (agreement) {Agreement};
        \node [property, below=of consensus, right=0.1cm of termination] (linearity) {Linearity};
        \node [property, below=of consensus, right=0.1cm of linearity] (responsiveness) {Responsiveness};

        \node [property, above=of smr] (liveness) {Liveness};
        \node [property, above=of smr, left=0.1cm of liveness] (safety) {Safety};
        \node [property, above=of smr, right=0.1cm of liveness, align=center] (to) {Total \\ ordering};

        \node [property, above=of blockchain, align=center] (cg) {Chain \\ growth};
        \node [property, above=of blockchain, left=0.1cm of cg, align=center] (cf) {Common prefix \\ (Consistency)};
        \node [property, above=of blockchain, right=0.1cm of cg, align=center] (cq) {Chain quality \\ (Fairness)};

        \draw[->,blue] (consensus) -- (smr);
        \draw[->,blue] (consensus) -- (blockchain);

        \draw[->,blue] (consensus) -- (agreement);
        \draw[->,blue] (consensus) -- (validity);
        \draw[->,blue] (consensus) -- (termination);
        \draw[->,blue] (consensus) -- (linearity);
        \draw[->,blue] (consensus) -- (responsiveness);

        \draw[->,blue] (smr) -- (safety);
        \draw[->,blue] (smr) -- (liveness);
        \draw[->,blue] (smr) -- (to);

        \draw[->,blue] (blockchain) -- (cf);
        \draw[->,blue] (blockchain) -- (cg);
        \draw[->,blue] (blockchain) -- (cq);


    \end{tikzpicture}

\end{frame}

%%%%%%%%%%%%%%%%%%%%%%%%%%%%%%%%%%%%%%%%%%%%%%%%%%%%%%%%%%%%%%%%%%%%%%%%%%%%%%%%%%%%%%%%%%%%%%%%

\section{Properties of Byzantine consensus protocols}


\begin{frame}
\frametitle{Consensus}

% https://disco.ethz.ch/courses/podc_allstars/lecture/chapter16.pdf

\begin{definition}[Consensus]
There are $n$ nodes, of which at most $f$ might be faulty (or Byzantine).
Each node $P_i$ starts with an input value (say $u_i$).
The nodes must decide one of those values (say $v_i$), satisfying three properties: Agreement, Validity and Termination.
\end{definition}

\end{frame}

\begin{frame}
    \frametitle{Properties of consensus protocols}

    Classic:

    \begin{itemize}
        \item Agreement
        \item Validity
        \item Termination
    \end{itemize}

    Emerging:

    \begin{itemize}
        \item Linearity
        \item Responsiveness
    \end{itemize}

\end{frame}

\begin{frame}
\frametitle{Properties of consensus protocols - Agreement}

(Informal) All correct processes must agree on the same value.

\begin{definition}[Agreement]
For any two honest players $P_i$ and $P_j$, $v_i = v_j$.
\end{definition}

\end{frame}


\begin{frame}
\frametitle{Properties of consensus protocols - Validity}

(Informal) The decision value must be the input value of a node.

\begin{definition}[Validity]
$\forall i: u_i \in \langle v_i \rangle$.
\end{definition}

\end{frame}


\begin{frame}
\frametitle{Properties of consensus protocols - Termination}

(Informal) All correct nodes terminate in finite time.

\begin{definition}[Termination]
All the honest players terminate with probability 1.
\end{definition}

\end{frame}



\begin{frame}
\frametitle{Properties of consensus protocols - Linearity}

First proposed in HotStuff~\cite{yin2019hotstuff}.

\begin{definition}[Linearity]
Any correct leader sends only $O(n)$ messages to drive a protocol to consensus.
\end{definition}



\end{frame}



\begin{frame}
\frametitle{Properties of consensus protocols - Responsiveness}

First defined in Hybrid Consensus~\cite{pass2017hybrid}.

(Informal) The transaction confirmation time depends only on the network’s actual delay, but not on any a-prior known upper-bound.

\begin{definition}[Responsiveness]
A consensus protocol is responsive if nodes can reach the consensus in time depending only on the network’s actual $\delta$ (message delays), not on the loose upper bound $\Delta$ (known upper bound on message delays).
\end{definition}

\end{frame}

%%%%%%%%%%%%%%%%%%%%%%%%%%%%%%%%%%%%%%%%%%%%%%%%%%%%%%%%%%%%%%%%%%%%%%%%%%%%%%%%%%%%%%%%%%%%%%%%


\section{Properties of state machine replication}


\begin{frame}
\frametitle{State machine replication}

State machine replication is a general paradigm for implementing fault-tolerant services by replicating servers and coordinating client interactions with server replicas~\cite{schneider1990implementing}.

A replicated state machine should:

\begin{enumerate}
    \item Place copies of the State Machine on multiple, independent servers.
    \item Receive client requests, interpreted as Inputs to the State Machine.
    \item Choose an ordering for the Inputs.
    \item Execute Inputs in the chosen order on each server.
    \item Respond to clients with the Output from the State Machine.
    \item Monitor replicas for differences in State or Output.
\end{enumerate}




\end{frame}

\begin{frame}
\frametitle{Properties of state machine replication}

\begin{itemize}
    \item Safety
    \item Liveness
    \item Total ordering
\end{itemize}

\end{frame}

\begin{frame}
\frametitle{Properties of state machine replication - Safety}

(Informal) When decisions are made by any two correct nodes, they decide on non-conflicting transactions.

\end{frame}

\begin{frame}
\frametitle{Properties of state machine replication - Liveness}

(Informal) $T$-Liveness: each honest node terminates and outputs a value at the end of $T$.

The value of $T$ depends on the research problems and protocols.

\end{frame}


\begin{frame}
\frametitle{Properties of state machine replication - Total ordering}

(Taken from \cite{duan2018beat}) 

\begin{definition}[Total ordering]
If a correct replica has delivered $m_1, m_2, \dots ,m_s$ and another correct replica has delivered $m'_1,m'_2, \dots,m'_{s'}$, then $m_i = m'_i$ for $1 \leq i \leq min(s, s')$.
\end{definition}

\end{frame}

%%%%%%%%%%%%%%%%%%%%%%%%%%%%%%%%%%%%%%%%%%%%%%%%%%%%%%%%%%%%%%%%%%%%%%%%%%%%%%%%%%%%%%%%%%%%%%%%

\section{Properties of blockchains}

\begin{frame}
\frametitle{Properties of blockchains}

\begin{itemize}
    \item Common prefix (Consistency)
    \item Chain growth
    \item Chain quality (Fairness)
\end{itemize}

\end{frame}


\begin{frame}
\frametitle{Properties of blockchains - Common prefix (Consistency)}

First proposed in~\cite{garay2015bitcoin}.

\begin{definition}[$k$-common-preifx]
For any pair of honest players $P_1$, $P_2$ adopting the chains $C_1$, $C_2$ at rounds $r_1 \leq r_2$, it holds that $\mathcal{C}_{1}^{\lceil k} \preceq \mathcal{C}_2$.
\end{definition}

\cite{pass2017analysis} refines Common Prefix to $T$-Consistency in order to provide a black-box reduction.

\begin{definition}[$T$-consistency]
With overwhelming probability, at any point, the chains of two honest players can differ only in the last $T$ blocks.
\end{definition}


\end{frame}

\begin{frame}
\frametitle{Properties of blockchains - Chain growth}

\begin{definition}[$(\tau, s)$-chain-growth]
For any honest party $P$ with chain $C$, it holds that for any $s$ rounds there are at least $\tau \cdot s$ blocks added to the chain of $P$.
\end{definition}

Some papers~\cite{rocket2018snowflake}\cite{pass2017sleepy} also define upper/lower bounds for chain growth.


\end{frame}


\begin{frame}
\frametitle{Properties of blockchains - Chain quality (Fairness)}
    
\begin{definition}[$(\mu, k)$-chain-quality]
The proportion of blocks in any $k$-long subsequence produced by the adversary is less than $\mu \cdot k$, where $\mu$ is the portion of mining power controlled by the adversary.
\end{definition}

\end{frame}


\begin{frame}[allowframebreaks]
    \frametitle{References}
    \bibliographystyle{alpha}
    \tiny\bibliography{refs}
\end{frame}

\end{document}