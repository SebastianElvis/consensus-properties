\documentclass{beamer}
\usetheme{Madrid}
 
\usepackage[utf8]{inputenc}

\usepackage{amsmath}
\usepackage{amsfonts}
\usepackage{array}

\usepackage{graphicx}
\graphicspath{{./figs/}}
\usepackage{subfigure}
\usepackage{multirow}
\usepackage{xspace}


\usepackage{tikz}
\usetikzlibrary{arrows, shapes, positioning}



%Information to be included in the title page:
\title{Properties of consensus protocols}

\author{Runchao Han}
\institute{}
\date{}

\begin{document}

\frame{\titlepage}

%%%%%%%%%%%%%%%%%%%%%%%%%%%%%%%%%%%%%%%%%%%%%%%%%%%%%%%%%%%%%%%%%%%%%%%%%%%%%%%%%%%%%%%%%%%%%%%%

\begin{frame}
    \frametitle{Summary}


    \tikzstyle{block} = [draw, rectangle]
    \tikzstyle{property} = [draw, ellipse, font=\tiny]


    \begin{tikzpicture}[]
        \node [block] (consensus) {Consensus};
        \node [block, above left=2cm of consensus, align=center] (smr) {State \\ machine \\ replication};
        \node [block, above right=2cm of consensus] (blockchain) {Blockchain};

        \node [property, below=of consensus] (termination) {Termination};
        \node [property, below=of consensus, left=0.1cm of termination] (validity) {Validity};
        \node [property, below=of consensus, left=0.1cm of validity] (agreement) {Agreement};
        \node [property, below=of consensus, right=0.1cm of termination] (linearity) {Linearity};
        \node [property, below=of consensus, right=0.1cm of linearity] (responsiveness) {Responsiveness};

        \node [property, above=of smr] (liveness) {Liveness};
        \node [property, above=of smr, left=0.1cm of liveness] (safety) {Safety};
        \node [property, above=of smr, right=0.1cm of liveness, align=center] (to) {Total \\ ordering};

        \node [property, above=of blockchain, align=center] (cg) {Chain \\ growth};
        \node [property, above=of blockchain, left=0.1cm of cg, align=center] (cf) {Common prefix \\ (Consistency)};
        \node [property, above=of blockchain, right=0.1cm of cg, align=center] (cq) {Chain quality \\ (Fairness)};
        \node [property, right=0.3cm of blockchain, align=center] (pers) {Persistence};
        \node [property, below=0.1 of pers, align=center] (liv) {Liveness};


        \draw[->,blue] (consensus) -- (smr);
        \draw[->,blue] (consensus) -- (blockchain);

        \draw[->,blue] (consensus) -- (agreement);
        \draw[->,blue] (consensus) -- (validity);
        \draw[->,blue] (consensus) -- (termination);
        \draw[->,blue] (consensus) -- (linearity);
        \draw[->,blue] (consensus) -- (responsiveness);

        \draw[->,blue] (smr) -- (safety);
        \draw[->,blue] (smr) -- (liveness);
        \draw[->,blue] (smr) -- (to);

        \draw[->,blue] (blockchain) -- (cf);
        \draw[->,blue] (blockchain) -- (cg);
        \draw[->,blue] (blockchain) -- (cq);
        \draw[->,blue] (blockchain) -- (pers);
        \draw[->,blue] (blockchain) -- (liv);



    \end{tikzpicture}

\end{frame}

%%%%%%%%%%%%%%%%%%%%%%%%%%%%%%%%%%%%%%%%%%%%%%%%%%%%%%%%%%%%%%%%%%%%%%%%%%%%%%%%%%%%%%%%%%%%%%%%

\section{Properties of Byzantine consensus protocols}


\begin{frame}
\frametitle{Consensus}

% https://disco.ethz.ch/courses/podc_allstars/lecture/chapter16.pdf

\begin{definition}[Consensus]
There are $n$ nodes, of which at most $f$ might be faulty (or Byzantine).
Each node $P_i$ starts with an input value (say $u_i$).
Each node $P_i$ must decide one of those values (say $v_i$).
\end{definition}

\end{frame}

\begin{frame}
    \frametitle{Properties of consensus protocols}

    Classic:

    \begin{itemize}
        \item Agreement
        \item Validity
        \item Termination
    \end{itemize}

    Emerging:

    \begin{itemize}
        \item Linearity
        \item Responsiveness
    \end{itemize}

\end{frame}

\begin{frame}
\frametitle{Properties of consensus protocols - Agreement}

All correct processes must agree on the same value.

\begin{definition}[Agreement]
For any two honest players $P_i$ and $P_j$, $v_i = v_j$.
\end{definition}

\end{frame}


\begin{frame}
\frametitle{Properties of consensus protocols - Validity}

The decision value must be the input value of a node.

\begin{definition}[Validity]
For all $i$, $v_i \in \langle u_i \rangle$.
\end{definition}

\end{frame}


\begin{frame}
\frametitle{Properties of consensus protocols - Termination}

All correct nodes terminate in finite time.

\begin{definition}[Termination]
Every honest player eventually decides some value.
\end{definition}

\end{frame}



\begin{frame}
\frametitle{Properties of consensus protocols - Linearity}

First defined in HotStuff~\cite{yin2019hotstuff}.

The communication complexity of the leader is linear.

\begin{definition}[Linearity]
Any correct leader sends only $O(n)$ messages to drive a protocol to consensus.
\end{definition}



\end{frame}



\begin{frame}
\frametitle{Properties of consensus protocols - Responsiveness}

First defined in Hybrid Consensus~\cite{pass2017hybrid}.

The time taken before termination depends only on the network delay, but not on any known upper-bound.

\begin{definition}[Responsiveness]
A consensus protocol is responsive if nodes can reach the consensus in time depending only on the network’s actual message delays, not on a known upper bound on message delays.
\end{definition}

\end{frame}

%%%%%%%%%%%%%%%%%%%%%%%%%%%%%%%%%%%%%%%%%%%%%%%%%%%%%%%%%%%%%%%%%%%%%%%%%%%%%%%%%%%%%%%%%%%%%%%%


\section{Properties of state machine replication}


\begin{frame}
\frametitle{State machine replication}

State machine replication is a general paradigm for implementing fault-tolerant services by replicating servers and coordinating client interactions with server replicas~\cite{schneider1990implementing}.

A replicated state machine should:

\begin{enumerate}
    \item Place copies of the State Machine on multiple, independent servers.
    \item Receive client requests, interpreted as Inputs to the State Machine.
    \item Choose an ordering for the Inputs.
    \item Execute Inputs in the chosen order on each server.
    \item Respond to clients with the Output from the State Machine.
    \item Monitor replicas for differences in State or Output.
\end{enumerate}




\end{frame}

\begin{frame}
\frametitle{Properties of state machine replication}

\begin{itemize}
    \item Safety
    \item Liveness
    \item Total ordering
\end{itemize}

Only total ordering has a unique formal definition.

How safety and liveness are defined is protocol-specific.

\end{frame}

\begin{frame}
\frametitle{Properties of state machine replication - Safety}

Bad things will never happen.

Example: 
\begin{itemize}
    \item When decisions are made by any two correct nodes, they decide on non-conflicting transactions.
\end{itemize}

\end{frame}

\begin{frame}
\frametitle{Properties of state machine replication - Liveness}

Good thins will eventually happen.

Example:
\begin{itemize}
    \item Each honest node terminates and outputs a value at the end of $T$ (a time bound).
\end{itemize}

\end{frame}


\begin{frame}
\frametitle{Properties of state machine replication - Total ordering}

(Taken from \cite{duan2018beat}) 

\begin{definition}[Total ordering]
If a correct replica has delivered $m_1, m_2, \dots ,m_s$ and another correct replica has delivered $m'_1,m'_2, \dots,m'_{s'}$, then $m_i = m'_i$ for $1 \leq i \leq min(s, s')$.
\end{definition}

\end{frame}

%%%%%%%%%%%%%%%%%%%%%%%%%%%%%%%%%%%%%%%%%%%%%%%%%%%%%%%%%%%%%%%%%%%%%%%%%%%%%%%%%%%%%%%%%%%%%%%%

\section{Properties of blockchains}

\begin{frame}
\frametitle{Properties of blockchains}

\begin{itemize}
    \item Correctness
    \begin{itemize}
        \item Common prefix (Consistency)
        \item Chain growth
        \item Chain quality (Fairness)
    \end{itemize}
    \item Robustness
    \begin{itemize}
        \item Persistence
        \item Liveness
    \end{itemize}
\end{itemize}

\end{frame}


\begin{frame}
\frametitle{Properties of blockchains - Cocrrectness - Common prefix (Consistency)}

First proposed in~\cite{garay2015bitcoin}.

\begin{definition}[$k$-common-preifx]
For any pair of honest players $P_1$, $P_2$ adopting the chains $C_1$, $C_2$ at rounds $r_1 \leq r_2$, it holds that $\mathcal{C}_{1}^{\lceil k} \preceq \mathcal{C}_2$.
\end{definition}

\cite{pass2017analysis} refines Common Prefix to $T$-Consistency in order to provide a black-box reduction.

\begin{definition}[$T$-consistency]
With overwhelming probability, at any point, the chains of two honest players can differ only in the last $T$ blocks.
\end{definition}


\end{frame}

\begin{frame}
\frametitle{Properties of blockchains - Cocrrectness - Chain growth}

\begin{definition}[$(\tau, s)$-chain-growth]
For any honest party $P$ with chain $C$, it holds that for any $s$ rounds there are at least $\tau \cdot s$ blocks added to the chain of $P$.
\end{definition}

Some papers~\cite{rocket2018snowflake}\cite{pass2017sleepy} also define upper/lower bounds for chain growth.


\end{frame}


\begin{frame}
\frametitle{Properties of blockchains - Cocrrectness - Chain quality (Fairness)}
    
\begin{definition}[$(\mu, k)$-chain-quality]
The proportion of blocks in any $k$-long subsequence produced by the adversary is less than $\mu \cdot k$, where $\mu$ is the portion of mining power controlled by the adversary.
\end{definition}

\end{frame}




\begin{frame}
    \frametitle{Properties of blockchains - Robustness - Persistence}

    First proposed in~\cite{garay2015bitcoin}.

    Given a tx, it is in the same position for blockchains of all honest nodes.

    \begin{definition}[Persistence]
        Parameterized by $k \in N$ (``depth'' parameter). If in a certain round an honest party reports a blockchain that contains a transaction $tx$ in a block at least $k$ blocks away from the end of the blockchain (such transaction will be called ``stable''), then whenever $tx$ is reported by any honest party it will be in the same position in the blockchain.
    \end{definition}

\end{frame}


\begin{frame}
    \frametitle{Properties of blockchains - Robustness - Liveness}
        
    First proposed in~\cite{garay2015bitcoin}.

    When all honest nodes commit a valid tx for a time period, there will be a honest node who includes it to the blockchain.

    \begin{definition}[Liveness]
        Parameterized by $u$ and $k$ (``waittime'' and ``depth'' parameter, resp.). A valid transaction is given as input to all honest parties of a blockchain continuously for the creation of $u$ consecutive blocks, then there exists an honest party who will report this transaction at a block more than $k$ blocks from the end of the ledger.
    \end{definition}

\end{frame}


\begin{frame}[allowframebreaks]
    \frametitle{References}
    \bibliographystyle{alpha}
    \tiny\bibliography{refs}
\end{frame}

\end{document}