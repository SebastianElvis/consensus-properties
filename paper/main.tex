\documentclass[runningheads]{llncs}
\usepackage{graphicx}

\begin{document}

\title{Properties of consensus protocols}

\author{Runchao Han}

\institute{
Monash University and CSIRO-Data61\\
\email{runchao.han@monash.edu}
}

\maketitle

% https://blockchainfoundations.org/docs/AK-Shanghai-2017.pdf

\section{Overview}



\section{Properties of consensus}

\subsection{Consensus}
$n$ machines (where $f$ machines are Byzantine) run a consensus protocol to agree on a single value.

\subsection{Classic properties}

\subsubsection{Agreement}
(Informal) All correct processes must agree on the same value.

\subsubsection{Validity}
(Informal) If all correct processes propose the same value, all processes must finally decide this value.

Sometimes also known as Integrity.

\subsubsection{Termination}
(Informal) Eventually, all correct processes decide some values.

\subsection{Emerging properties}

\subsubsection{Linearity}
Proposed in HotStuff~\cite{yin2019hotstuff}.

(Informal) The communication complexity of consensus is $O(n)$ where $n$ is the number of participants.

\subsubsection{Responsiveness}
Proposed in Hybrid Consensus~\cite{pass2017hybrid}.

(Informal) The transaction confirmation time depends only on the network’s actual delay, but not on any a-prior known upper-bound.




\section{Properties of state machine replication}

\subsection{State machine replication}
Replicate a machine to $n$ machines to run a single service that can tolerate $\geq f$ Byzantine machines ($n > f$).

\subsection{Properties}

\subsubsection{Safety}
(Informal) When decisions are made by any two correct nodes, they decide on non-conflicting transactions.

\subsubsection{Liveness}
(Informal) $T$-Liveness: each honest node terminates and outputs a value at the end of $T$.

The value of $T$ depends on the research problems and protocols...






\section{Properties of blockchains}

\subsection{Classic}

\subsubsection{Common prefix and $T$-Consistency}
$k$-common-preifx~\cite{garay2015bitcoin}:
for any pair of honest players $P_1$, $P_2$ adopting the chains $C_1$, $C_2$ at rounds $r_1 \leq r_2$, it holds that $\mathcal{C}_{1}^{\lceil k} \preceq \mathcal{C}_2$.

\cite{pass2017analysis} refines common prefix to consistency in order to provide a black-box reduction.

$T$-consistency~\cite{pass2017analysis}:
with overwhelming probability, at any point, the chains of two honest players can differ only in the last $T$ blocks.

\subsubsection{Chain growth}
$(\tau, s)$-chain-growth:
for any honest party $P$ with chain $C$, it holds that for any $s$ rounds there are at least $\tau \cdot s$ blocks added to the chain of $P$.

Some papers~\cite{rocket2018snowflake}~\cite{pass2017sleepy} also define upper/lower bounds for chain growth.

\subsubsection{Chain quality}
$(\mu, k)$-chain-quality: the proportion of blocks in any $k$-long subsequence produced by the adversary is less than $\mu \cdot k$, where $\mu$ is the portion of mining power controlled by the adversary.




\bibliographystyle{splncs04}
\bibliography{refs}

\end{document}
